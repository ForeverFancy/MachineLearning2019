\textbf{Notice, }to get the full credits, please present your solutions step by step.

\begin{exercise}[Linear regression \textnormal{20pts}]
	Given a data set $\{ (x_i ,y_i) \}_{i=1}^{n}$, where $x_i,y_i\in \mathbb{R}$. 
	\begin{enumerate}
	    \item If we want to fit the data by a linear model
	        \begin{align}\label{eqn:linear}
	            y =  w_0 + w_1 x,
	        \end{align}
	        please find $\hat{w}_0$ and $\hat{w}_1$ by the least squares approach (you need to find expressions of $\hat{w}_0$ and $\hat{w}_1$ by $\{ (x_i ,y_i) \}_{i=1}^{n}$, respectively).
	    \item \textbf{Programming Exercise} We provide you a data set $\{ (x_i ,y_i) \}_{i=1}^{30}$. Consider the model in (\ref{eqn:linear}) and the one as follows:
	        \begin{align}\label{eqn:linear-quadratic}
	            y =  w_0 + w_1 x+ w_2 x^2. 
	        \end{align}
	        Which model do you think fits better the data? Please detail your approach first and then implement it by your favorite programming language. The required output includes 
	        \begin{enumerate}
	            \item your detailed approach step by step; 
	            \item your code with detailed comments according to your planned approach; 
	            \item a plot showing the data and the fitting models; 
	            \item the model you finally choose [$\hat{w}_0$ and $\hat{w}_1$ if you choose the model in (\ref{eqn:linear}), or $\hat{w}_0$, $\hat{w}_1$, and $\hat{w}_2$ if you choose the model in (\ref{eqn:linear})].
	        \end{enumerate}
	\end{enumerate}
\end{exercise}
\newpage
\begin{solution}
	\heiti
	\ \\
	\begin{enumerate}
		\item 令误差函数\\$E = \sum\limits_{i=1}^n(w_0 + w_1x_i -y_i)^2$
	
		$(w_0^*,w_1^*) = \arg\min\limits_{(w_0,w_1)}(E)$
	
		$\frac{\partial E}{\partial w_0} = 2\sum\limits_{i=1}^n(w_0 + w_1x_i - y_i)$
	
		$\frac{\partial E}{\partial w_1} = 2\sum\limits_{i=1}^nx_i(w_0 + w_1x_i - y_i)$
	
		要使得 E 最小,所以令两个偏导数为 0,那么可得
	
		$nw_0 = \sum\limits_{i=1}^n(w_1x_i-y_i) $
	
		$w_0\sum\limits_{i=1}^nx_i = n\sum\limits_{i=1}^nx_i(w_1x_i-y_i)$
		
		联立方程解得
	
		$ w_0 = \frac{n\sum\limits_{i=1}^ny_ix_i - \sum\limits_{i=1}^nx_i\sum\limits_{i=1}^ny_i}{n\sum\limits_{i=1}^nx_i^2-(\sum\limits_{i=1}^nx_i)^2}$
	
		$ w_1 = \frac{\sum\limits_{i=1}^n(y_i-w_0x_i)}{n}$ 
		\item \ 
			\begin{enumerate}
				\item 主要思路是由 1. 中已经求得的 $w0,w1$ 表达式可以通过样本直接计算出对应的 $w0,w1$ ,再重新代入误差函数 $E$ 中,求出最小误差。\\
				同样的道理,二次拟合函数也可以通过计算其误差函数,对三个参数分别求偏导数并令其为 0,联立三个方程解得相应的三个参数,并代回到误差函数中求得最小误差。\\
				以上两个最小误差进行比较即可看出哪个模型对数据的拟合较好。\\
				具体到编程细节,首先读入数据并转换为 numpy array,分别进行线性拟合和二次拟合(其中二次拟合使用 numpy 求解线性方程组),求得最小误差后进行比较,并绘制图像。\\
				参考文献:\href{https://www.bb.ustc.edu.cn/jpkc/xiaoji/szjsff/jsffkj/chapt3_2.htm}{计算方法课程} 
				\item 代码见附件\href{./prob1.py}{prob1.py}。
				\item 图片见附件\href{./Figure.png}{Figure.png}。
				\item 最终比较之后选择二次拟合模型,得到 $w0 = 1.0295683746564661, w1 = 0.3861433340323225, w2 = -0.14215111308616024$。
			\end{enumerate}
	\end{enumerate}
\end{solution}

\begin{exercise}[Rank of matrices \textnormal{20pts}]
	Let $\mathbf{A} \in \mathbb{R}^{m\times n}$ and $\mathbf{B}\in \mathbb{R}^{n\times p}$.
	\begin{enumerate}
	    \item Please show that
            \begin{enumerate}
                \item $\rank{\mathbf{A}} = \rank{\mathbf{A}^{\top}}$;
                \item $\rank{\mathbf{A}\mathbf{B}} \leq \rank{\mathbf{A}}$;
                \item $\rank{\mathbf{A}\mathbf{B}} \leq \rank{\mathbf{B}}$;
                \item $\rank{\mathbf{A}} = \rank{\mathbf{A}^{\top}  \mathbf{A}}$.
            \end{enumerate}
        \item The \emph{column space} of $\mathbf{A}$ is defined by
                \begin{align*}
                    \mathcal{C}(\mathbf{A} ) = \{ \mathbf{y}\in \mathbb{R}^m : \mathbf{y} = \mathbf{Ax},\,\mathbf{x}\in\mathbb{R}^n\}.
                \end{align*}
                The \emph{null space} of $\mathbf{A}$ is defined by
                \begin{align*}
                    \mathcal{N}(\mathbf{A})  = \{ \mathbf{x}\in \mathbb{R}^n : \mathbf{Ax}=0\}.
                \end{align*}
                Notice that, the rank of $\mathbf{A}$ is the dimension of the column space of $\mathbf{A}$.
                
                Please show that:
	               \begin{enumerate}
                	    \item $\rank{\mathbf{A}} + \dim ( \mathcal{N}( \mathbf{A} ) ) = n$;
                	    \item let $\mathbf{y}\in \mathbb{R}^m$, show that $\mathbf{y}=0$ if and only if $\mathbf{a}_i^{\top}\mathbf{y}=0$ for $i=1,\ldots,m$, where $\{\mathbf{a}_1,\mathbf{a}_2,\ldots,\mathbf{a}_m\}$ is a basis of $\mathbb{R}^m$.
                	\end{enumerate}    
	\end{enumerate}
\end{exercise}

\begin{solution}
	\heiti
	\ \\
	\begin{enumerate}
		\item \ \\
		\begin{enumerate}
			\item 设$ rank(A) = r$,那么存在可逆方阵 P 和 Q 使得 $PAQ = \begin{pmatrix} I_r & O \\ O & O \\ \end{pmatrix}$,两边同时转置可得 $ Q^TA^TP^T = \begin{pmatrix} I_r & O\\ O & O \\ \end{pmatrix} $,于是根据定义可得 $ rank(A^T) = r = rank(A)$。
			\item 设$ rank(A) = r, rank(B) = s$,那么根据定义可得存在可逆方阵 $ P_1, Q_1, P_2, Q_2$,使得$ A = P_1 \begin{pmatrix} I_r & O \\ O & O \\ \end{pmatrix} Q_1, B = P_2 \begin{pmatrix} I_s & O \\ O & O \\ \end{pmatrix} Q_2$,于是 $ AB = P_1 \begin{pmatrix} I_r & O \\ O & O \\ \end{pmatrix} Q_1 P_2 \begin{pmatrix} I_s & O \\ O & O \\ \end{pmatrix} Q_2$,我们可以设 $Q1P2 = \begin{pmatrix} R1 & R2 \\ R3 & R4 \end{pmatrix}$,则 $rank(AB) = rank(P1\begin{pmatrix} R_1 & O \\ O & O \end{pmatrix}Q2)$,其中 $R1$ 是一个大小为 ${r \times s}$ 的矩阵。于是 $rank(AB) \leq min(r, s) = min(rank(A), rank(B))$,所以 $ rank(AB) \leq rank(A), rank(AB) \leq rank(B)$。
			\item 已在 b. 中证明。
			\item 证明需要使用 2 中的 null space,也就是方程 $ Ax=0 $ 的解空间。考虑 
			\begin{align*}
			\mathcal{N}(\mathbf{A})  = \{ \mathbf{x}\in \mathbb{R}^n : \mathbf{Ax}=0\}.
			\end{align*}
			当 $x \in \mathcal{N}(\mathbf{A})$ 时,有
			\begin{align*}
				Ax = 0\\
			A^TAx = 0\\
			 \Rightarrow x \in \mathcal{N}(\mathbf{A^TA})\\
			 \Rightarrow \mathcal{N}(\mathbf{A}) \in \mathcal{N}(\mathbf{A^TA})
			\end{align*}
			当 $x \in \mathcal{N}(\mathbf{A^TA})$ 时,有
			\begin{align*}
			A^TAx = 0\\
			x^TA^TAx = 0\\
			\Rightarrow (Ax)^TAx = 0\\
			\Rightarrow Ax = 0\\
			\Rightarrow x \in \mathcal{N}(\mathbf{A})\\
			\Rightarrow \mathcal{N}(\mathbf{A^TA}) \in \mathcal{N}(\mathbf{A})
			\end{align*}
			所以有
			\begin{align*}
			\mathcal{N}(\mathbf{A^TA}) = \mathcal{N}(\mathbf{A})\\
			\Rightarrow dim(\mathcal{N}(\mathbf{A^TA})) = dim(\mathcal{N}(\mathbf{A}))\\
			\Rightarrow n - dim(\mathcal{N}(\mathbf{A^TA})) = n - dim(\mathcal{N}(\mathbf{A}))\\
			\Rightarrow rank(A^TA) = rank(A)
			\end{align*}
			其中利用了
			\begin{align*}
			rank(A) = n - dim(dim(\mathcal{N}(\mathbf{A}))\\
			rank(A^TA) = n - dim(dim(\mathcal{N}(\mathbf{A^TA})).\\
			\end{align*}
		\end{enumerate}
		\item \ \\
		\begin{enumerate}
			\item 首先设 $rank(A) = r$,可以对线性方程组 $Ax = 0$ 做初等行变换,变为最简形式 $ Jx = 0$,并且 $rank(A) = rank(J) = r$,最后矩阵可以变为 r 个线性无关的方程组成的方程组,解可以表示为 $ t_1, t_2\dots t_{n - r} $这 $n - r$ 个自由变量的线性组合\\
			$\left\{\begin{array}{l}{x_{1}=\alpha_{11} t_{1}+\alpha_{12} t_{2}+\cdots+\alpha_{1, n-r} t_{n-r}} \\ {x_{2}=\alpha_{21} t_{1}+\alpha_{22} t_{2}+\cdots+\alpha_{2, n-r} t_{n-r}} \\ {\ldots \ldots \ldots \ldots} \\ {x_{m}=\alpha_{m 1} t_{1}+\alpha_{m 2} t_{2}+\cdots+\alpha_{m, n-r} t_{n-r}}\end{array}\right.$\\
			写成向量形式
			\begin{align*}
				\mathbf{x} = t_{1} \mathbf{\alpha_{1}}+\cdots+t_{n-r} \mathbf{ \alpha_{n-r}}
			\end{align*}	
			下面要证明 $\mathbf{\alpha_{1}}+\cdots+t_{n-r} \mathbf{ \alpha_{n-r}}$ 是线性无关的,可以注意到 $\mathbf{\alpha_{1}}+\cdots+t_{n-r} \mathbf{ \alpha_{n-r}}$ 也是线性方程组的通解,因此 $\mathbf{x} = t_{1} \mathbf{\alpha_{1}}+\cdots+t_{n-r} \mathbf{ \alpha_{n-r}}$ 中必定有 $ n - r$ 个分量分别为 $ t_{1} \mathbf{\alpha_{1}}+\cdots+t_{n-r} \mathbf{ \alpha_{n-r}}$,所以如果
			\begin{align*}
				t_{1} \mathbf{\alpha_{1}}+\cdots+t_{n-r} \mathbf{ \alpha_{n-r}} = \mathbf{0}
			\end{align*}
			那么必有 $t_{1} = \cdots = t_{n-r} = 0$,所以 $ \mathbf{\alpha_{1}} , \cdots , \mathbf{ \alpha_{n-r}} $线性无关,这说明$ \mathbf{\alpha_{1}} , \cdots , \mathbf{ \alpha_{n-r}} $是解空间,即$\mathcal{N}(\mathbf{A})$ 的一组基,而且 $ dim(\mathcal{N}(\mathbf{A})) = n - r$,所以 $\rank{\mathbf{A}} + \dim(\mathcal{N}(\mathbf{A} ) ) = n$.
			\item 易知若 $ y = 0 $,那么对任意 $\mathbf{a}_i \in \{\mathbf{a}_1,\mathbf{a}_2,\ldots,\mathbf{a}_m\}$,,均有 $\mathbf{a}_i^{\top}\mathbf{y} = 0$. 反之,若对任意 $\mathbf{a}_i \in \{\mathbf{a}_1,\mathbf{a}_2,\ldots,\mathbf{a}_m\}$,,均有 $\mathbf{a}_i^{\top}\mathbf{y} = 0$,那么有 $\begin{pmatrix}
				a_1 & a_2 & \dots & a_m
			\end{pmatrix}^{\mathbf{T}} y = 0 $,假设 $ y \neq 0$,那么 $y = \begin{pmatrix}
				y1 & y2 & \dots & y_m
			\end{pmatrix}$,且至少存在一个分量不为 0 ,将矩阵乘法展开可知 $\mathbf{a}_1y_1 + \mathbf{a}_2y_2 + \dots + \mathbf{a}_my_m = 0$,由于 $\{\mathbf{a}_1,\mathbf{a}_2,\ldots,\mathbf{a}_m\}$ 是线性空间的一组基,根据定义一定有 $y_1 = y_2 = \dots = y_m = 0$,这与假设矛盾,所以 $ y = 0$.
		\end{enumerate}
	\end{enumerate}

\end{solution}

\newpage


\newpage

\begin{exercise}[ \textnormal{5pts}] 
	Let $\mathbf{x}\in \mathbf{R}^n$. Find the gradients of the following functions.
	\begin{enumerate}
	    \item $f(\mathbf{x}) = \mathbf{a}^{\top}\mathbf{x}$.
	    \item $f(\mathbf{x}) = \mathbf{x}^{\top}\mathbf{x}$.
	    \item $f(\mathbf{x})=\| \mathbf{y} - \mathbf{A}\mathbf{x} \|_2^2$, where $\mathbf{A}\in\mathbb{R}^{m\times n}$.
	\end{enumerate}
\end{exercise}

\begin{solution}
	\heiti
	\ \\
	\begin{enumerate}
		\item $\frac{\mathbf{d}f(\mathbf{x})}{\mathbf{d}x} = \mathbf{a}$
		\item $\frac{\mathbf{d}f(\mathbf{x})}{\mathbf{d}x} = 2\mathbf{x}$
		\item $\frac{\mathbf{d}f(\mathbf{x})}{\mathbf{d}x} = \frac{\mathbf{d}f(\mathbf{x})}{\mathbf{d}(\mathbf{y} - \mathbf{A}\mathbf{x})} \frac{\mathbf{d}(\mathbf{y} - \mathbf{A}\mathbf{x})}{\mathbf{d}x} = - 2(\mathbf{y} - \mathbf{A}\mathbf{x})\mathbf{A}^\mathbf{T}$
	\end{enumerate}
\end{solution}
\newpage


\begin{exercise}[Second-order sufficient optimality conditions \textnormal{10pts}]
	Suppose that $f:\mathbb{R}^n\rightarrow\mathbb{R}$ is twice differentiable at $\mathbf{x}$. If $\nabla f(\mathbf{x})=0$ and the Hessian matrix $\mathbf{H}(\mathbf{x})$ is positive definite, then $\mathbf{x}$ is a strict local minimum.
	\begin{enumerate}
		\item Show the above result by contradiction.
		\item Show the result by NOT using contradiction. [\emph{Hint: you may need eigen-decomposition.}]
	\end{enumerate}
\end{exercise}

\begin{solution}

\end{solution}
\newpage

\begin{exercise}[Identically independently distributed \textnormal{10pts}]
	Suppose that the training samples $\{(\mathbf{x}_i,y_i)\}_{i=1}^n$ are i.i.d.. show that
	\begin{align*}
	    p(\mathbf{x}_1,\ldots,\mathbf{x}_n)=\prod_{i=1}^np(\mathbf{x}_i).
	\end{align*}
	
\end{exercise}

\begin{solution}
	\heiti
	\ \\
	根据独立的定义,n 个样本相互独立,那么它们同时发生的概率等于分别发生的概率乘积,即
	\begin{align*}
	    p(\mathbf{x}_1,\ldots,\mathbf{x}_n)=\prod_{i=1}^np(\mathbf{x}_i).
	\end{align*}
	于是得证。
\end{solution}
\newpage


\begin{exercise}[First-order condition \RNum{2} \textnormal{5pts}]
	Suppose that $f$ is continuously differentiable. Prove that $f$ is convex if and only if $\dom f$ is convex and
	\begin{align*}
	    \langle \nabla f(\mathbf{x}) - \nabla f(\mathbf{y}) , \mathbf{x} - \mathbf{y} \rangle \geq 0.
	\end{align*}
	
\end{exercise}

\begin{solution}

\end{solution}