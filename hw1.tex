%%%%%%%%%%%%%%%%%%%%%%%%%%%%%%%%%%%%%%%%%%%%%%%%%%%%%%%%%%%%%%%%
%
%  Template for homework of Introduction to Machine Learning.
%
%  Fill in your name, lecture number, lecture date and body
%  of homework as indicated below.
%
%%%%%%%%%%%%%%%%%%%%%%%%%%%%%%%%%%%%%%%%%%%%%%%%%%%%%%%%%%%%%%%%


\documentclass[11pt,letter,notitlepage]{article}
%Mise en page
\usepackage[left=2cm, right=2cm, lines=45, top=0.8in, bottom=0.7in]{geometry}
\usepackage{fancyhdr}
\usepackage{fancybox}
\usepackage{graphicx}
\usepackage{pdfpages} 
\usepackage{enumitem}
\usepackage[UTF8]{ctex}
\usepackage[colorlinks,linkcolor=blue]{hyperref}
\renewcommand{\headrulewidth}{1.5pt}
\renewcommand{\footrulewidth}{1.5pt}
\pagestyle{fancy}
\newcommand\Loadedframemethod{TikZ}
\usepackage[framemethod=\Loadedframemethod]{mdframed}

\usepackage{amssymb,amsmath}
\usepackage{amsthm}
\usepackage{thmtools}

\setlength{\topmargin}{0pt}
\setlength{\textheight}{9in}
\setlength{\headheight}{0pt}

\setlength{\oddsidemargin}{0.25in}
\setlength{\textwidth}{6in}

%%%%%%%%%%%%%%%%%%%%%%%%
%% Define the Exercise environment %%
%%%%%%%%%%%%%%%%%%%%%%%%
\mdtheorem[
topline=false,
rightline=false,
leftline=false,
bottomline=false,
leftmargin=-10,
rightmargin=-10
]{exercise}{\textbf{Exercise}}
%%%%%%%%%%%%%%%%%%%%%%%
%% End of the Exercise environment %%
%%%%%%%%%%%%%%%%%%%%%%%

%%%%%%%%%%%%%%%%%%%%%%%
%% Define the Solution Environment %%
%%%%%%%%%%%%%%%%%%%%%%%
\declaretheoremstyle
[
spaceabove=0pt, 
spacebelow=0pt, 
headfont=\normalfont\bfseries,
notefont=\mdseries, 
notebraces={(}{)}, 
headpunct={:\quad}, 
headindent={},
postheadspace={ }, 
postheadspace=4pt, 
bodyfont=\normalfont, 
qed=$\blacksquare$,
preheadhook={\begin{mdframed}[style=myframedstyle]},
	postfoothook=\end{mdframed},
]{mystyle}

\declaretheorem[style=mystyle,title=Solution,numbered=no]{solution}
\mdfdefinestyle{myframedstyle}{%
	topline=false,
	rightline=false,
	leftline=false,
	bottomline=false,
	skipabove=-6ex,
	leftmargin=-10,
	rightmargin=-10}
%%%%%%%%%%%%%%%%%%%%%%%
%% End of the Solution environment %%
%%%%%%%%%%%%%%%%%%%%%%%

%% Homework info.
\newcommand{\posted}{\text{Seq. 20, 2019}}       			%%% FILL IN POST DATE HERE
\newcommand{\due}{\text{Seq. 30, 2019}} 			%%% FILL IN Due DATE HERE
\newcommand{\hwno}{\text{1}} 		           			%%% FILL IN LECTURE NUMBER HERE


%%%%%%%%%%%%%%%%%%%%
%% Put your information here %%
%%%%%%%%%%%%%%%%%%%
\newcommand{\name}{\text{Bowen Zhang}}  	          			%%% FILL IN YOUR NAME HERE
\newcommand{\id}{\text{PB17000215}}		       			%%% FILL IN YOUR ID HERE
%%%%%%%%%%%%%%%%%%%%
%% End of the student's info %%
%%%%%%%%%%%%%%%%%%%


\newcommand{\proj}[2]{\textbf{P}_{#2} (#1)}
\newcommand{\lspan}[1]{\textbf{span}  (#1)  }
\newcommand{\rank}[1]{ \textbf{rank}  (#1)  }
\newcommand{\dom}{ \textbf{dom}  }
\newcommand{\RNum}[1]{\uppercase\expandafter{\romannumeral #1\relax}}


\lhead{
	\textbf{\name}
}
\rhead{
	\textbf{\id}
}
\chead{\textbf{
		Homework \hwno
}}


\begin{document}
\vspace*{-4\baselineskip}
\thispagestyle{empty}


\begin{center}
{\bf\large Introduction to Machine Learning}\\
{Fall 2019}\\
University of Science and Technology of China
\end{center}

\noindent
Lecturer: Jie Wang  			 %%% FILL IN LECTURER HERE
\hfill
Homework \hwno             			
\\
Posted: \posted
\hfill
Due: \due
\\
Name: \name             			
\hfill
ID: \id						
\hfill

\noindent
\rule{\textwidth}{2pt}

\medskip





%%%%%%%%%%%%%%%%%%%%%%%%%%%%%%%%%%%%%%%%%%%%%%%%%%%%%%%%%%%%%%%%
%% BODY OF HOMEWORK GOES HERE
%%%%%%%%%%%%%%%%%%%%%%%%%%%%%%%%%%%%%%%%%%%%%%%%%%%%%%%%%%%%%%%%

\begin{exercise}[Rank of matrices \textnormal{20pts}]
	Let $\mathbf{A} \in \mathbb{R}^{m\times n}$ and $\mathbf{B}\in \mathbb{R}^{n\times p}$.
	\begin{enumerate}
	    \item Please show that
            \begin{enumerate}
                \item $\rank{\mathbf{A}} = \rank{\mathbf{A}^{\top}}$;
                \item $\rank{\mathbf{A}\mathbf{B}} \leq \rank{\mathbf{A}}$;
                \item $\rank{\mathbf{A}\mathbf{B}} \leq \rank{\mathbf{B}}$;
                \item $\rank{\mathbf{A}} = \rank{\mathbf{A}^{\top}  \mathbf{A}}$.
            \end{enumerate}
        \item The \emph{column space} of $\mathbf{A}$ is defined by
                \begin{align*}
                    \mathcal{C}(\mathbf{A} ) = \{ \mathbf{y}\in \mathbb{R}^m : \mathbf{y} = \mathbf{Ax},\,\mathbf{x}\in\mathbb{R}^n\}.
                \end{align*}
                The \emph{null space} of $\mathbf{A}$ is defined by
                \begin{align*}
                    \mathcal{N}(\mathbf{A})  = \{ \mathbf{x}\in \mathbb{R}^n : \mathbf{Ax}=0\}.
                \end{align*}
                Notice that, the rank of $\mathbf{A}$ is the dimension of the column space of $\mathbf{A}$.
                
                Please show that:
	               \begin{enumerate}
                	    \item $\rank{\mathbf{A}} + \dim ( \mathcal{N}( \mathbf{A} ) ) = n$;
                	    \item let $\mathbf{y}\in \mathbb{R}^m$, show that $\mathbf{y}=0$ if and only if $\mathbf{a}_i^{\top}\mathbf{y}=0$ for $i=1,\ldots,m$, where $\{\mathbf{a}_1,\mathbf{a}_2,\ldots,\mathbf{a}_m\}$ is a basis of $\mathbb{R}^m$.
                	\end{enumerate}    
	\end{enumerate}
\end{exercise}

\begin{solution}

\end{solution}

%%%%%%%%%%%%%%%%%%%%%%%%%%%%%%%%%%%%%%%%%%%%%%%%%%%%%%%%%%%%%%%%

\end{document}
