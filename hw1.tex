%%%%%%%%%%%%%%%%%%%%%%%%%%%%%%%%%%%%%%%%%%%%%%%%%%%%%%%%%%%%%%%%
%
%  Template for homework of Introduction to Machine Learning.
%
%  Fill in your name, lecture number, lecture date and body
%  of homework as indicated below.
%
%%%%%%%%%%%%%%%%%%%%%%%%%%%%%%%%%%%%%%%%%%%%%%%%%%%%%%%%%%%%%%%%


\documentclass[11pt,letter,notitlepage]{article}
%Mise en page
\usepackage[left=2cm, right=2cm, lines=45, top=0.8in, bottom=0.7in]{geometry}
\usepackage{fancyhdr}
\usepackage{fancybox}
\usepackage{graphicx}
\usepackage{pdfpages} 
\usepackage{enumitem}
\usepackage[UTF8]{ctex}
\usepackage[colorlinks,linkcolor=blue]{hyperref}
\renewcommand{\headrulewidth}{1.5pt}
\renewcommand{\footrulewidth}{1.5pt}
\pagestyle{fancy}
\newcommand\Loadedframemethod{TikZ}
\usepackage[framemethod=\Loadedframemethod]{mdframed}

\usepackage{amssymb,amsmath}
\usepackage{amsthm}
\usepackage{thmtools}

\setlength{\topmargin}{0pt}
\setlength{\textheight}{9in}
\setlength{\headheight}{0pt}

\setlength{\oddsidemargin}{0.25in}
\setlength{\textwidth}{6in}

%%%%%%%%%%%%%%%%%%%%%%%%
%% Define the Exercise environment %%
%%%%%%%%%%%%%%%%%%%%%%%%
\mdtheorem[
topline=false,
rightline=false,
leftline=false,
bottomline=false,
leftmargin=-10,
rightmargin=-10
]{exercise}{\textbf{Exercise}}
%%%%%%%%%%%%%%%%%%%%%%%
%% End of the Exercise environment %%
%%%%%%%%%%%%%%%%%%%%%%%

%%%%%%%%%%%%%%%%%%%%%%%
%% Define the Solution Environment %%
%%%%%%%%%%%%%%%%%%%%%%%
\declaretheoremstyle
[
spaceabove=0pt, 
spacebelow=0pt, 
headfont=\normalfont\bfseries,
notefont=\mdseries, 
notebraces={(}{)}, 
headpunct={:\quad}, 
headindent={},
postheadspace={ }, 
postheadspace=4pt, 
bodyfont=\normalfont, 
qed=$\blacksquare$,
preheadhook={\begin{mdframed}[style=myframedstyle]},
	postfoothook=\end{mdframed},
]{mystyle}

\declaretheorem[style=mystyle,title=Solution,numbered=no]{solution}
\mdfdefinestyle{myframedstyle}{%
	topline=false,
	rightline=false,
	leftline=false,
	bottomline=false,
	skipabove=-6ex,
	leftmargin=-10,
	rightmargin=-10}
%%%%%%%%%%%%%%%%%%%%%%%
%% End of the Solution environment %%
%%%%%%%%%%%%%%%%%%%%%%%

%% Homework info.
\newcommand{\posted}{\text{Seq. 20, 2019}}       			%%% FILL IN POST DATE HERE
\newcommand{\due}{\text{Seq. 30, 2019}} 			%%% FILL IN Due DATE HERE
\newcommand{\hwno}{\text{1}} 		           			%%% FILL IN LECTURE NUMBER HERE


%%%%%%%%%%%%%%%%%%%%
%% Put your information here %%
%%%%%%%%%%%%%%%%%%%
\newcommand{\name}{\text{Bowen Zhang}}  	          			%%% FILL IN YOUR NAME HERE
\newcommand{\id}{\text{PB17000215}}		       			%%% FILL IN YOUR ID HERE
%%%%%%%%%%%%%%%%%%%%
%% End of the student's info %%
%%%%%%%%%%%%%%%%%%%


\newcommand{\proj}[2]{\textbf{P}_{#2} (#1)}
\newcommand{\lspan}[1]{\textbf{span}  (#1)  }
\newcommand{\rank}[1]{ \textbf{rank}  (#1)  }
\newcommand{\dom}{ \textbf{dom}  }
\newcommand{\RNum}[1]{\uppercase\expandafter{\romannumeral #1\relax}}


\lhead{
	\textbf{\name}
}
\rhead{
	\textbf{\id}
}
\chead{\textbf{
		Homework \hwno
}}


\begin{document}
\vspace*{-4\baselineskip}
\thispagestyle{empty}


\begin{center}
{\bf\large Introduction to Machine Learning}\\
{Fall 2019}\\
University of Science and Technology of China
\end{center}

\noindent
Lecturer: Jie Wang  			 %%% FILL IN LECTURER HERE
\hfill
Homework \hwno             			
\\
Posted: \posted
\hfill
Due: \due
\\
Name: \name             			
\hfill
ID: \id						
\hfill

\noindent
\rule{\textwidth}{2pt}

\medskip





%%%%%%%%%%%%%%%%%%%%%%%%%%%%%%%%%%%%%%%%%%%%%%%%%%%%%%%%%%%%%%%%
%% BODY OF HOMEWORK GOES HERE
%%%%%%%%%%%%%%%%%%%%%%%%%%%%%%%%%%%%%%%%%%%%%%%%%%%%%%%%%%%%%%%%

\begin{exercise}[Projection \textnormal{30pts}]
	Let $C \subset \mathbb{R}^n$ be a closed convex set and $\mathbf{x} \in \mathbb{R}^n$. Define
	\begin{align*}
	    \proj{\mathbf{x}}{C} = \arg\min_{\mathbf{y} \in C}\| \mathbf{y} - \mathbf{x} \|_2.    
	\end{align*}
    We call $\proj{\mathbf{x}}{C}$ the projection of the point $\mathbf{x}$ onto the convex set $C$. 
    \begin{enumerate}
        \item Show that any finite dimensional space is convex.
        \item Let $\mathbf{v}_i \in \mathbb{R}^n$, $i=1,\ldots,d$ with $d\leq n$, which are linearly independent.
            \begin{enumerate}
		        \item For any $\mathbf{w}\in \mathbb{R}^n$, please find $\proj{\mathbf{w}}{\mathbf{v}_1}$, which is the projection of $\mathbf{w}$ onto the subspace spanned by $\mathbf{v}_1$.  
		        \item Please show $\proj{\cdot}{\mathbf{v}_1}$ is a linear map, i.e.,
		            \begin{align*}
		                \proj{\alpha\mathbf{u}+\beta\mathbf{w}}{\mathbf{v}_1}=\alpha\proj{\mathbf{u}}{\mathbf{v}_1} + \beta \proj{\mathbf{w}}{\mathbf{v}_1},
		            \end{align*}
		            where $\alpha,\beta\in\mathbb{R}$ and $\mathbf{w}\in\mathbb{R}^n$.
		        \item Please find the projection matrix corresponding to the linear map $\proj{\cdot}{\mathbf{v}_1}$, i.e., find the matrix $\mathbf{H}_1\in\mathbb{R}^{n\times n}$ such that
		            \begin{align*}
		                \proj{\mathbf{w}}{\mathbf{v}_1}=\mathbf{H}_1\mathbf{w}.
		            \end{align*}
		        \item Let $\mathbf{V}=(\mathbf{v}_1,\ldots,\mathbf{v}_d)$. 
		            \begin{enumerate}
		                \item For any $\mathbf{w}\in \mathbb{R}^n$, please find              $\proj{\mathbf{w}}{\mathbf{V}}$, which is the projection of $\mathbf{w}$ onto $\mathcal{C}(\mathbf{V})$, and the corresponding projection matrix $\mathbf{H}$. 
		                \item Please find $\mathbf{H}$ if we further assume that $\mathbf{v}_i^{\top}\mathbf{v}_j=0$, $\forall\,i\neq j$.
		            \end{enumerate}
	        \end{enumerate}
	   \item A matrix $\mathbf{P}$ is called a projection matrix if $\mathbf{P}\mathbf{x}$ is the projection of $\mathbf{x}$ onto $\mathcal{C}(\mathbf{P})$ for any $\mathbf{x}$.
	        \begin{enumerate}
	            \item Let $\lambda$ be the eigenvalue of $\mathbf{P}$. Show that $\lambda$ is either $1$ or $0$. (\emph{Hint: you may want to figure out what are the eigenspaces corresponding to $\lambda=1$ and $\lambda=0$, respectively.})
	            \item Show that $\mathbf{P}$ is a projection matrix if and only if $\mathbf{P}^2 = \mathbf{P}$.
	        \end{enumerate}
	\end{enumerate}
\end{exercise}

\begin{solution}
	\heiti
	\ \\
	\begin{enumerate}
		\item 有限维空间一般指有限维线性空间,对空间中的任意向量 $\mathbf{x}, \mathbf{y}$,都有 $\alpha\mathbf{x} + (1-\alpha)\mathbf{y}$ 仍然在空间中(线性空间定义),所以根据定义,任意有限维空间都是凸的。
		\item \ \\
		\begin{enumerate}
			\item 
		\end{enumerate}
	\end{enumerate}
\end{solution}

\begin{exercise}[First-order condition \RNum{2} \textnormal{5pts}]
	Suppose that $f$ is continuously differentiable. Prove that $f$ is convex if and only if $\dom f$ is convex and
	\begin{align*}
	    \langle \nabla f(\mathbf{x}) - \nabla f(\mathbf{y}) , \mathbf{x} - \mathbf{y} \rangle \geq 0.
	\end{align*}
	
\end{exercise}

\begin{solution}

\end{solution}

%%%%%%%%%%%%%%%%%%%%%%%%%%%%%%%%%%%%%%%%%%%%%%%%%%%%%%%%%%%%%%%%

\end{document}
