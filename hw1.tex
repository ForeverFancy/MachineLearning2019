%%%%%%%%%%%%%%%%%%%%%%%%%%%%%%%%%%%%%%%%%%%%%%%%%%%%%%%%%%%%%%%%
%
%  Template for homework of Introduction to Machine Learning.
%
%  Fill in your name, lecture number, lecture date and body
%  of homework as indicated below.
%
%%%%%%%%%%%%%%%%%%%%%%%%%%%%%%%%%%%%%%%%%%%%%%%%%%%%%%%%%%%%%%%%


\documentclass[11pt,letter,notitlepage]{article}
%Mise en page
\usepackage[left=2cm, right=2cm, lines=45, top=0.8in, bottom=0.7in]{geometry}
\usepackage{fancyhdr}
\usepackage{fancybox}
\usepackage{graphicx}
\usepackage{pdfpages} 
\usepackage{enumitem}
\usepackage[UTF8]{ctex}
\usepackage[colorlinks,linkcolor=blue]{hyperref}
\renewcommand{\headrulewidth}{1.5pt}
\renewcommand{\footrulewidth}{1.5pt}
\pagestyle{fancy}
\newcommand\Loadedframemethod{TikZ}
\usepackage[framemethod=\Loadedframemethod]{mdframed}

\usepackage{amssymb,amsmath}
\usepackage{amsthm}
\usepackage{thmtools}

\setlength{\topmargin}{0pt}
\setlength{\textheight}{9in}
\setlength{\headheight}{0pt}

\setlength{\oddsidemargin}{0.25in}
\setlength{\textwidth}{6in}

%%%%%%%%%%%%%%%%%%%%%%%%
%% Define the Exercise environment %%
%%%%%%%%%%%%%%%%%%%%%%%%
\mdtheorem[
topline=false,
rightline=false,
leftline=false,
bottomline=false,
leftmargin=-10,
rightmargin=-10
]{exercise}{\textbf{Exercise}}
%%%%%%%%%%%%%%%%%%%%%%%
%% End of the Exercise environment %%
%%%%%%%%%%%%%%%%%%%%%%%

%%%%%%%%%%%%%%%%%%%%%%%
%% Define the Solution Environment %%
%%%%%%%%%%%%%%%%%%%%%%%
\declaretheoremstyle
[
spaceabove=0pt, 
spacebelow=0pt, 
headfont=\normalfont\bfseries,
notefont=\mdseries, 
notebraces={(}{)}, 
headpunct={:\quad}, 
headindent={},
postheadspace={ }, 
postheadspace=4pt, 
bodyfont=\normalfont, 
qed=$\blacksquare$,
preheadhook={\begin{mdframed}[style=myframedstyle]},
	postfoothook=\end{mdframed},
]{mystyle}

\declaretheorem[style=mystyle,title=Solution,numbered=no]{solution}
\mdfdefinestyle{myframedstyle}{%
	topline=false,
	rightline=false,
	leftline=false,
	bottomline=false,
	skipabove=-6ex,
	leftmargin=-10,
	rightmargin=-10}
%%%%%%%%%%%%%%%%%%%%%%%
%% End of the Solution environment %%
%%%%%%%%%%%%%%%%%%%%%%%

%% Homework info.
\newcommand{\posted}{\text{Seq. 20, 2019}}       			%%% FILL IN POST DATE HERE
\newcommand{\due}{\text{Seq. 30, 2019}} 			%%% FILL IN Due DATE HERE
\newcommand{\hwno}{\text{1}} 		           			%%% FILL IN LECTURE NUMBER HERE


%%%%%%%%%%%%%%%%%%%%
%% Put your information here %%
%%%%%%%%%%%%%%%%%%%
\newcommand{\name}{\text{Bowen Zhang}}  	          			%%% FILL IN YOUR NAME HERE
\newcommand{\id}{\text{PB17000215}}		       			%%% FILL IN YOUR ID HERE
%%%%%%%%%%%%%%%%%%%%
%% End of the student's info %%
%%%%%%%%%%%%%%%%%%%


\newcommand{\proj}[2]{\textbf{P}_{#2} (#1)}
\newcommand{\lspan}[1]{\textbf{span}  (#1)  }
\newcommand{\rank}[1]{ \textbf{rank}  (#1)  }
\newcommand{\dom}{ \textbf{dom}  }
\newcommand{\RNum}[1]{\uppercase\expandafter{\romannumeral #1\relax}}


\lhead{
	\textbf{\name}
}
\rhead{
	\textbf{\id}
}
\chead{\textbf{
		Homework \hwno
}}


\begin{document}
\vspace*{-4\baselineskip}
\thispagestyle{empty}


\begin{center}
{\bf\large Introduction to Machine Learning}\\
{Fall 2019}\\
University of Science and Technology of China
\end{center}

\noindent
Lecturer: Jie Wang  			 %%% FILL IN LECTURER HERE
\hfill
Homework \hwno             			
\\
Posted: \posted
\hfill
Due: \due
\\
Name: \name             			
\hfill
ID: \id						
\hfill

\noindent
\rule{\textwidth}{2pt}

\medskip





%%%%%%%%%%%%%%%%%%%%%%%%%%%%%%%%%%%%%%%%%%%%%%%%%%%%%%%%%%%%%%%%
%% BODY OF HOMEWORK GOES HERE
%%%%%%%%%%%%%%%%%%%%%%%%%%%%%%%%%%%%%%%%%%%%%%%%%%%%%%%%%%%%%%%%

\begin{exercise}[Rank of matrices \textnormal{20pts}]
	Let $\mathbf{A} \in \mathbb{R}^{m\times n}$ and $\mathbf{B}\in \mathbb{R}^{n\times p}$.
	\begin{enumerate}
	    \item Please show that
            \begin{enumerate}
                \item $\rank{\mathbf{A}} = \rank{\mathbf{A}^{\top}}$;
                \item $\rank{\mathbf{A}\mathbf{B}} \leq \rank{\mathbf{A}}$;
                \item $\rank{\mathbf{A}\mathbf{B}} \leq \rank{\mathbf{B}}$;
                \item $\rank{\mathbf{A}} = \rank{\mathbf{A}^{\top}  \mathbf{A}}$.
            \end{enumerate}
        \item The \emph{column space} of $\mathbf{A}$ is defined by
                \begin{align*}
                    \mathcal{C}(\mathbf{A} ) = \{ \mathbf{y}\in \mathbb{R}^m : \mathbf{y} = \mathbf{Ax},\,\mathbf{x}\in\mathbb{R}^n\}.
                \end{align*}
                The \emph{null space} of $\mathbf{A}$ is defined by
                \begin{align*}
                    \mathcal{N}(\mathbf{A})  = \{ \mathbf{x}\in \mathbb{R}^n : \mathbf{Ax}=0\}.
                \end{align*}
                Notice that, the rank of $\mathbf{A}$ is the dimension of the column space of $\mathbf{A}$.
                
                Please show that:
	               \begin{enumerate}
                	    \item $\rank{\mathbf{A}} + \dim ( \mathcal{N}( \mathbf{A} ) ) = n$;
                	    \item let $\mathbf{y}\in \mathbb{R}^m$, show that $\mathbf{y}=0$ if and only if $\mathbf{a}_i^{\top}\mathbf{y}=0$ for $i=1,\ldots,m$, where $\{\mathbf{a}_1,\mathbf{a}_2,\ldots,\mathbf{a}_m\}$ is a basis of $\mathbb{R}^m$.
                	\end{enumerate}    
	\end{enumerate}
\end{exercise}

\begin{solution}
	\heiti
	\ \\
	\begin{enumerate}
		\item \ \\
		\begin{enumerate}
			\item 设$ rank(A) = r$,那么存在可逆方阵 P 和 Q 使得 $PAQ = \begin{pmatrix} I_r & O \\ O & O \\ \end{pmatrix}$,两边同时转置可得 $ Q^TA^TP^T = \begin{pmatrix} I_r & O\\ O & O \\ \end{pmatrix} $,于是根据定义可得 $ rank(A^T) = r = rank(A)$。
			\item 设$ rank(A) = r, rank(B) = s$,那么根据定义可得存在可逆方阵 $ P_1, Q_1, P_2, Q_2$,使得$ A = P_1 \begin{pmatrix} I_r & O \\ O & O \\ \end{pmatrix} Q_1, B = P_2 \begin{pmatrix} I_s & O \\ O & O \\ \end{pmatrix} Q_2$,于是 $ AB = P_1 \begin{pmatrix} I_r & O \\ O & O \\ \end{pmatrix} Q_1 P_2 \begin{pmatrix} I_s & O \\ O & O \\ \end{pmatrix} Q_2$,我们可以设 $Q1P2 = \begin{pmatrix} R1 & R2 \\ R3 & R4 \end{pmatrix}$,则 $rank(AB) = rank(P1\begin{pmatrix} R_1 & O \\ O & O \end{pmatrix}Q2)$,其中 $R1$ 是一个大小为 ${r \times s}$ 的矩阵。于是 $rank(AB) \leq min(r, s) = min(rank(A), rank(B))$,所以 $ rank(AB) \leq rank(A), rank(AB) \leq rank(B)$。
			\item 已在 b. 中证明。
			\item 证明需要使用 2 中的 null space,也就是方程 $ Ax=0 $ 的解空间。考虑 
			\begin{align*}
			\mathcal{N}(\mathbf{A})  = \{ \mathbf{x}\in \mathbb{R}^n : \mathbf{Ax}=0\}.
			\end{align*}
			当 $x \in \mathcal{N}(\mathbf{A})$ 时,有
			\begin{align*}
				Ax = 0\\
			A^TAx = 0\\
			 \Rightarrow x \in \mathcal{N}(\mathbf{A^TA})\\
			 \Rightarrow \mathcal{N}(\mathbf{A}) \in \mathcal{N}(\mathbf{A^TA})
			\end{align*}
			当 $x \in \mathcal{N}(\mathbf{A^TA})$ 时,有
			\begin{align*}
			A^TAx = 0\\
			x^TA^TAx = 0\\
			\Rightarrow (Ax)^TAx = 0\\
			\Rightarrow Ax = 0\\
			\Rightarrow x \in \mathcal{N}(\mathbf{A})\\
			\Rightarrow \mathcal{N}(\mathbf{A^TA}) \in \mathcal{N}(\mathbf{A})
			\end{align*}
			所以有
			\begin{align*}
			\mathcal{N}(\mathbf{A^TA}) = \mathcal{N}(\mathbf{A})\\
			\Rightarrow dim(\mathcal{N}(\mathbf{A^TA})) = dim(\mathcal{N}(\mathbf{A}))\\
			\Rightarrow n - dim(\mathcal{N}(\mathbf{A^TA})) = n - dim(\mathcal{N}(\mathbf{A}))\\
			\Rightarrow rank(A^TA) = rank(A)
			\end{align*}
			其中利用了
			\begin{align*}
			rank(A) = n - dim(dim(\mathcal{N}(\mathbf{A}))\\
			rank(A^TA) = n - dim(dim(\mathcal{N}(\mathbf{A^TA})).\\
			\end{align*}
		\end{enumerate}
		\item \ \\
		\begin{enumerate}
			\item 首先设 $rank(A) = r$,可以对线性方程组 $Ax = 0$ 做初等行变换,变为最简形式 $ Jx = 0$,并且 $rank(A) = rank(J) = r$,最后矩阵可以变为 r 个线性无关的方程组成的方程组,解可以表示为 $ t_1, t_2\dots t_{n - r} $这 $n - r$ 个自由变量的线性组合\\
			$\left\{\begin{array}{l}{x_{1}=\alpha_{11} t_{1}+\alpha_{12} t_{2}+\cdots+\alpha_{1, n-r} t_{n-r}} \\ {x_{2}=\alpha_{21} t_{1}+\alpha_{22} t_{2}+\cdots+\alpha_{2, n-r} t_{n-r}} \\ {\ldots \ldots \ldots \ldots} \\ {x_{m}=\alpha_{m 1} t_{1}+\alpha_{m 2} t_{2}+\cdots+\alpha_{m, n-r} t_{n-r}}\end{array}\right.$\\
			写成向量形式
			\begin{align*}
				\mathbf{x} = t_{1} \mathbf{\alpha_{1}}+\cdots+t_{n-r} \mathbf{ \alpha_{n-r}}
			\end{align*}	
			下面要证明 $\mathbf{\alpha_{1}}+\cdots+t_{n-r} \mathbf{ \alpha_{n-r}}$ 是线性无关的,可以注意到 $\mathbf{\alpha_{1}}+\cdots+t_{n-r} \mathbf{ \alpha_{n-r}}$ 也是线性方程组的通解,因此 $\mathbf{x} = t_{1} \mathbf{\alpha_{1}}+\cdots+t_{n-r} \mathbf{ \alpha_{n-r}}$ 中必定有 $ n - r$ 个分量分别为 $ t_{1} \mathbf{\alpha_{1}}+\cdots+t_{n-r} \mathbf{ \alpha_{n-r}}$,所以如果
			\begin{align*}
				t_{1} \mathbf{\alpha_{1}}+\cdots+t_{n-r} \mathbf{ \alpha_{n-r}} = \mathbf{0}
			\end{align*}
			那么必有 $t_{1} = \cdots = t_{n-r} = 0$,所以 $ \mathbf{\alpha_{1}} , \cdots , \mathbf{ \alpha_{n-r}} $线性无关,这说明$ \mathbf{\alpha_{1}} , \cdots , \mathbf{ \alpha_{n-r}} $是解空间,即$\mathcal{N}(\mathbf{A})$ 的一组基,而且 $ dim(\mathcal{N}(\mathbf{A})) = n - r$,所以 $\rank{\mathbf{A}} + \dim(\mathcal{N}(\mathbf{A} ) ) = n$.
			\item 易知若 $ y = 0 $,那么对任意 $\mathbf{a}_i \in \{\mathbf{a}_1,\mathbf{a}_2,\ldots,\mathbf{a}_m\}$,,均有 $\mathbf{a}_i^{\top}\mathbf{y} = 0$. 反之,若对任意 $\mathbf{a}_i \in \{\mathbf{a}_1,\mathbf{a}_2,\ldots,\mathbf{a}_m\}$,,均有 $\mathbf{a}_i^{\top}\mathbf{y} = 0$,那么有 $\begin{pmatrix}
				a_1 & a_2 & \dots & a_m
			\end{pmatrix}^{\mathbf{T}} y = 0 $,假设 $ y \neq 0$,那么 $y = \begin{pmatrix}
				y1 & y2 & \dots & y_m
			\end{pmatrix}$,且至少存在一个分量不为 0 ,将矩阵乘法展开可知 $\mathbf{a}_1y_1 + \mathbf{a}_2y_2 + \dots + \mathbf{a}_my_m = 0$,由于 $\{\mathbf{a}_1,\mathbf{a}_2,\ldots,\mathbf{a}_m\}$ 是线性空间的一组基,根据定义一定有 $y_1 = y_2 = \dots = y_m = 0$,这与假设矛盾,所以 $ y = 0$.
		\end{enumerate}
	\end{enumerate}

\end{solution}

%%%%%%%%%%%%%%%%%%%%%%%%%%%%%%%%%%%%%%%%%%%%%%%%%%%%%%%%%%%%%%%%

\end{document}
