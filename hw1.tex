%%%%%%%%%%%%%%%%%%%%%%%%%%%%%%%%%%%%%%%%%%%%%%%%%%%%%%%%%%%%%%%%
%
%  Template for homework of Introduction to Machine Learning.
%
%  Fill in your name, lecture number, lecture date and body
%  of homework as indicated below.
%
%%%%%%%%%%%%%%%%%%%%%%%%%%%%%%%%%%%%%%%%%%%%%%%%%%%%%%%%%%%%%%%%


\documentclass[11pt,letter,notitlepage]{article}
%Mise en page
\usepackage[left=2cm, right=2cm, lines=45, top=0.8in, bottom=0.7in]{geometry}
\usepackage{fancyhdr}
\usepackage{fancybox}
\usepackage{graphicx}
\usepackage{pdfpages} 
\usepackage{enumitem}
\usepackage[UTF8]{ctex}
\usepackage[colorlinks,linkcolor=blue]{hyperref}
\renewcommand{\headrulewidth}{1.5pt}
\renewcommand{\footrulewidth}{1.5pt}
\pagestyle{fancy}
\newcommand\Loadedframemethod{TikZ}
\usepackage[framemethod=\Loadedframemethod]{mdframed}

\usepackage{amssymb,amsmath}
\usepackage{amsthm}
\usepackage{thmtools}

\setlength{\topmargin}{0pt}
\setlength{\textheight}{9in}
\setlength{\headheight}{0pt}

\setlength{\oddsidemargin}{0.25in}
\setlength{\textwidth}{6in}

%%%%%%%%%%%%%%%%%%%%%%%%
%% Define the Exercise environment %%
%%%%%%%%%%%%%%%%%%%%%%%%
\mdtheorem[
topline=false,
rightline=false,
leftline=false,
bottomline=false,
leftmargin=-10,
rightmargin=-10
]{exercise}{\textbf{Exercise}}
%%%%%%%%%%%%%%%%%%%%%%%
%% End of the Exercise environment %%
%%%%%%%%%%%%%%%%%%%%%%%

%%%%%%%%%%%%%%%%%%%%%%%
%% Define the Solution Environment %%
%%%%%%%%%%%%%%%%%%%%%%%
\declaretheoremstyle
[
spaceabove=0pt, 
spacebelow=0pt, 
headfont=\normalfont\bfseries,
notefont=\mdseries, 
notebraces={(}{)}, 
headpunct={:\quad}, 
headindent={},
postheadspace={ }, 
postheadspace=4pt, 
bodyfont=\normalfont, 
qed=$\blacksquare$,
preheadhook={\begin{mdframed}[style=myframedstyle]},
	postfoothook=\end{mdframed},
]{mystyle}

\declaretheorem[style=mystyle,title=Solution,numbered=no]{solution}
\mdfdefinestyle{myframedstyle}{%
	topline=false,
	rightline=false,
	leftline=false,
	bottomline=false,
	skipabove=-6ex,
	leftmargin=-10,
	rightmargin=-10}
%%%%%%%%%%%%%%%%%%%%%%%
%% End of the Solution environment %%
%%%%%%%%%%%%%%%%%%%%%%%

%% Homework info.
\newcommand{\posted}{\text{Seq. 20, 2019}}       			%%% FILL IN POST DATE HERE
\newcommand{\due}{\text{Seq. 30, 2019}} 			%%% FILL IN Due DATE HERE
\newcommand{\hwno}{\text{1}} 		           			%%% FILL IN LECTURE NUMBER HERE


%%%%%%%%%%%%%%%%%%%%
%% Put your information here %%
%%%%%%%%%%%%%%%%%%%
\newcommand{\name}{\text{Bowen Zhang}}  	          			%%% FILL IN YOUR NAME HERE
\newcommand{\id}{\text{PB17000215}}		       			%%% FILL IN YOUR ID HERE
%%%%%%%%%%%%%%%%%%%%
%% End of the student's info %%
%%%%%%%%%%%%%%%%%%%


\newcommand{\proj}[2]{\textbf{P}_{#2} (#1)}
\newcommand{\lspan}[1]{\textbf{span}  (#1)  }
\newcommand{\rank}[1]{ \textbf{rank}  (#1)  }
\newcommand{\dom}{ \textbf{dom}  }
\newcommand{\RNum}[1]{\uppercase\expandafter{\romannumeral #1\relax}}


\lhead{
	\textbf{\name}
}
\rhead{
	\textbf{\id}
}
\chead{\textbf{
		Homework \hwno
}}


\begin{document}
\vspace*{-4\baselineskip}
\thispagestyle{empty}


\begin{center}
{\bf\large Introduction to Machine Learning}\\
{Fall 2019}\\
University of Science and Technology of China
\end{center}

\noindent
Lecturer: Jie Wang  			 %%% FILL IN LECTURER HERE
\hfill
Homework \hwno             			
\\
Posted: \posted
\hfill
Due: \due
\\
Name: \name             			
\hfill
ID: \id						
\hfill

\noindent
\rule{\textwidth}{2pt}

\medskip





%%%%%%%%%%%%%%%%%%%%%%%%%%%%%%%%%%%%%%%%%%%%%%%%%%%%%%%%%%%%%%%%
%% BODY OF HOMEWORK GOES HERE
%%%%%%%%%%%%%%%%%%%%%%%%%%%%%%%%%%%%%%%%%%%%%%%%%%%%%%%%%%%%%%%%

\textbf{Notice, }to get the full credits, please present your solutions step by step.

\begin{exercise}[Linear regression \textnormal{20pts}]
	Given a data set $\{ (x_i ,y_i) \}_{i=1}^{n}$, where $x_i,y_i\in \mathbb{R}$. 
	\begin{enumerate}
	    \item If we want to fit the data by a linear model
	        \begin{align}\label{eqn:linear}
	            y =  w_0 + w_1 x,
	        \end{align}
	        please find $\hat{w}_0$ and $\hat{w}_1$ by the least squares approach (you need to find expressions of $\hat{w}_0$ and $\hat{w}_1$ by $\{ (x_i ,y_i) \}_{i=1}^{n}$, respectively).
	    \item \textbf{Programming Exercise} We provide you a data set $\{ (x_i ,y_i) \}_{i=1}^{30}$. Consider the model in (\ref{eqn:linear}) and the one as follows:
	        \begin{align}\label{eqn:linear-quadratic}
	            y =  w_0 + w_1 x+ w_2 x^2. 
	        \end{align}
	        Which model do you think fits better the data? Please detail your approach first and then implement it by your favorite programming language. The required output includes 
	        \begin{enumerate}
	            \item your detailed approach step by step; 
	            \item your code with detailed comments according to your planned approach; 
	            \item a plot showing the data and the fitting models; 
	            \item the model you finally choose [$\hat{w}_0$ and $\hat{w}_1$ if you choose the model in (\ref{eqn:linear}), or $\hat{w}_0$, $\hat{w}_1$, and $\hat{w}_2$ if you choose the model in (\ref{eqn:linear})].
	        \end{enumerate}
	\end{enumerate}
\end{exercise}
\newpage
\begin{solution}
	\heiti
	\ \\
	\begin{enumerate}
		\item 令误差函数\\$E = \sum\limits_{i=1}^n(w_0 + w_1x_i -y_i)^2$
	
		$(w_0^*,w_1^*) = \arg\min\limits_{(w_0,w_1)}(E)$
	
		$\frac{\partial E}{\partial w_0} = 2\sum\limits_{i=1}^n(w_0 + w_1x_i - y_i)$
	
		$\frac{\partial E}{\partial w_1} = 2\sum\limits_{i=1}^nx_i(w_0 + w_1x_i - y_i)$
	
		要使得 E 最小,所以令两个偏导数为 0,那么可得
	
		$nw_0 = \sum\limits_{i=1}^n(w_1x_i-y_i) $
	
		$w_0\sum\limits_{i=1}^nx_i = n\sum\limits_{i=1}^nx_i(w_1x_i-y_i)$
		
		联立方程解得
	
		$ w_0 = \frac{n\sum\limits_{i=1}^ny_ix_i - \sum\limits_{i=1}^nx_i\sum\limits_{i=1}^ny_i}{n\sum\limits_{i=1}^nx_i^2-(\sum\limits_{i=1}^nx_i)^2}$
	
		$ w_1 = \frac{\sum\limits_{i=1}^n(y_i-w_0x_i)}{n}$ 
		\item \ 
			\begin{enumerate}
				\item 主要思路是由 1. 中已经求得的 $w0,w1$ 表达式可以通过样本直接计算出对应的 $w0,w1$ ,再重新代入误差函数 $E$ 中,求出最小误差。\\
				同样的道理,二次拟合函数也可以通过计算其误差函数,对三个参数分别求偏导数并令其为 0,联立三个方程解得相应的三个参数,并代回到误差函数中求得最小误差。\\
				以上两个最小误差进行比较即可看出哪个模型对数据的拟合较好。\\
				具体到编程细节,首先读入数据并转换为 numpy array,分别进行线性拟合和二次拟合(其中二次拟合使用 numpy 求解线性方程组),求得最小误差后进行比较,并绘制图像。\\
				参考文献:\href{https://www.bb.ustc.edu.cn/jpkc/xiaoji/szjsff/jsffkj/chapt3_2.htm}{计算方法课程} 
				\item 代码见附件\href{./prob1.py}{prob1.py}。
				\item 图片见附件\href{./Figure.png}{Figure.png}。
				\item 最终比较之后选择二次拟合模型,得到 $w0 = 1.0295683746564661, w1 = 0.3861433340323225, w2 = -0.14215111308616024$。
			\end{enumerate}
	\end{enumerate}
\end{solution}

\begin{exercise}[Rank of matrices \textnormal{20pts}]
	Let $\mathbf{A} \in \mathbb{R}^{m\times n}$ and $\mathbf{B}\in \mathbb{R}^{n\times p}$.
	\begin{enumerate}
	    \item Please show that
            \begin{enumerate}
                \item $\rank{\mathbf{A}} = \rank{\mathbf{A}^{\top}}$;
                \item $\rank{\mathbf{A}\mathbf{B}} \leq \rank{\mathbf{A}}$;
                \item $\rank{\mathbf{A}\mathbf{B}} \leq \rank{\mathbf{B}}$;
                \item $\rank{\mathbf{A}} = \rank{\mathbf{A}^{\top}  \mathbf{A}}$.
            \end{enumerate}
        \item The \emph{column space} of $\mathbf{A}$ is defined by
                \begin{align*}
                    \mathcal{C}(\mathbf{A} ) = \{ \mathbf{y}\in \mathbb{R}^m : \mathbf{y} = \mathbf{Ax},\,\mathbf{x}\in\mathbb{R}^n\}.
                \end{align*}
                The \emph{null space} of $\mathbf{A}$ is defined by
                \begin{align*}
                    \mathcal{N}(\mathbf{A})  = \{ \mathbf{x}\in \mathbb{R}^n : \mathbf{Ax}=0\}.
                \end{align*}
                Notice that, the rank of $\mathbf{A}$ is the dimension of the column space of $\mathbf{A}$.
                
                Please show that:
	               \begin{enumerate}
                	    \item $\rank{\mathbf{A}} + \dim ( \mathcal{N}( \mathbf{A} ) ) = n$;
                	    \item let $\mathbf{y}\in \mathbb{R}^m$, show that $\mathbf{y}=0$ if and only if $\mathbf{a}_i^{\top}\mathbf{y}=0$ for $i=1,\ldots,m$, where $\{\mathbf{a}_1,\mathbf{a}_2,\ldots,\mathbf{a}_m\}$ is a basis of $\mathbb{R}^m$.
                	\end{enumerate}    
	\end{enumerate}
\end{exercise}

\begin{solution}
	\heiti
	\ \\
	\begin{enumerate}
		\item \ \\
		\begin{enumerate}
			\item 设$ rank(A) = r$,那么存在可逆方阵 P 和 Q 使得 $PAQ = \begin{pmatrix} I_r & O \\ O & O \\ \end{pmatrix}$,两边同时转置可得 $ Q^TA^TP^T = \begin{pmatrix} I_r & O\\ O & O \\ \end{pmatrix} $,于是根据定义可得 $ rank(A^T) = r = rank(A)$。
			\item 设$ rank(A) = r, rank(B) = s$,那么根据定义可得存在可逆方阵 $ P_1, Q_1, P_2, Q_2$,使得$ A = P_1 \begin{pmatrix} I_r & O \\ O & O \\ \end{pmatrix} Q_1, B = P_2 \begin{pmatrix} I_s & O \\ O & O \\ \end{pmatrix} Q_2$,于是 $ AB = P_1 \begin{pmatrix} I_r & O \\ O & O \\ \end{pmatrix} Q_1 P_2 \begin{pmatrix} I_s & O \\ O & O \\ \end{pmatrix} Q_2$,我们可以设 $Q1P2 = \begin{pmatrix} R1 & R2 \\ R3 & R4 \end{pmatrix}$,则 $rank(AB) = rank(P1\begin{pmatrix} R_1 & O \\ O & O \end{pmatrix}Q2)$,其中 $R1$ 是一个大小为 ${r \times s}$ 的矩阵。于是 $rank(AB) \leq min(r, s) = min(rank(A), rank(B))$,所以 $ rank(AB) \leq rank(A), rank(AB) \leq rank(B)$。
			\item 已在 b. 中证明。
			\item 证明需要使用 2 中的 null space,也就是方程 $ Ax=0 $ 的解空间。考虑 
			\begin{align*}
			\mathcal{N}(\mathbf{A})  = \{ \mathbf{x}\in \mathbb{R}^n : \mathbf{Ax}=0\}.
			\end{align*}
			当 $x \in \mathcal{N}(\mathbf{A})$ 时,有
			\begin{align*}
				Ax = 0\\
			A^TAx = 0\\
			 \Rightarrow x \in \mathcal{N}(\mathbf{A^TA})\\
			 \Rightarrow \mathcal{N}(\mathbf{A}) \in \mathcal{N}(\mathbf{A^TA})
			\end{align*}
			当 $x \in \mathcal{N}(\mathbf{A^TA})$ 时,有
			\begin{align*}
			A^TAx = 0\\
			x^TA^TAx = 0\\
			\Rightarrow (Ax)^TAx = 0\\
			\Rightarrow Ax = 0\\
			\Rightarrow x \in \mathcal{N}(\mathbf{A})\\
			\Rightarrow \mathcal{N}(\mathbf{A^TA}) \in \mathcal{N}(\mathbf{A})
			\end{align*}
			所以有
			\begin{align*}
			\mathcal{N}(\mathbf{A^TA}) = \mathcal{N}(\mathbf{A})\\
			\Rightarrow dim(\mathcal{N}(\mathbf{A^TA})) = dim(\mathcal{N}(\mathbf{A}))\\
			\Rightarrow n - dim(\mathcal{N}(\mathbf{A^TA})) = n - dim(\mathcal{N}(\mathbf{A}))\\
			\Rightarrow rank(A^TA) = rank(A)
			\end{align*}
			其中利用了
			\begin{align*}
			rank(A) = n - dim(dim(\mathcal{N}(\mathbf{A}))\\
			rank(A^TA) = n - dim(dim(\mathcal{N}(\mathbf{A^TA})).\\
			\end{align*}
		\end{enumerate}
		\item \ \\
		\begin{enumerate}
			\item 首先设 $rank(A) = r$,可以对线性方程组 $Ax = 0$ 做初等行变换,变为最简形式 $ Jx = 0$,并且 $rank(A) = rank(J) = r$,最后矩阵可以变为 r 个线性无关的方程组成的方程组,解可以表示为 $ t_1, t_2\dots t_{n - r} $这 $n - r$ 个自由变量的线性组合\\
			$\left\{\begin{array}{l}{x_{1}=\alpha_{11} t_{1}+\alpha_{12} t_{2}+\cdots+\alpha_{1, n-r} t_{n-r}} \\ {x_{2}=\alpha_{21} t_{1}+\alpha_{22} t_{2}+\cdots+\alpha_{2, n-r} t_{n-r}} \\ {\ldots \ldots \ldots \ldots} \\ {x_{m}=\alpha_{m 1} t_{1}+\alpha_{m 2} t_{2}+\cdots+\alpha_{m, n-r} t_{n-r}}\end{array}\right.$\\
			写成向量形式
			\begin{align*}
				\mathbf{x} = t_{1} \mathbf{\alpha_{1}}+\cdots+t_{n-r} \mathbf{ \alpha_{n-r}}
			\end{align*}	
			下面要证明 $\mathbf{\alpha_{1}}+\cdots+t_{n-r} \mathbf{ \alpha_{n-r}}$ 是线性无关的,可以注意到 $\mathbf{\alpha_{1}}+\cdots+t_{n-r} \mathbf{ \alpha_{n-r}}$ 也是线性方程组的通解,因此 $\mathbf{x} = t_{1} \mathbf{\alpha_{1}}+\cdots+t_{n-r} \mathbf{ \alpha_{n-r}}$ 中必定有 $ n - r$ 个分量分别为 $ t_{1} \mathbf{\alpha_{1}}+\cdots+t_{n-r} \mathbf{ \alpha_{n-r}}$,所以如果
			\begin{align*}
				t_{1} \mathbf{\alpha_{1}}+\cdots+t_{n-r} \mathbf{ \alpha_{n-r}} = \mathbf{0}
			\end{align*}
			那么必有 $t_{1} = \cdots = t_{n-r} = 0$,所以 $ \mathbf{\alpha_{1}} , \cdots , \mathbf{ \alpha_{n-r}} $线性无关,这说明$ \mathbf{\alpha_{1}} , \cdots , \mathbf{ \alpha_{n-r}} $是解空间,即$\mathcal{N}(\mathbf{A})$ 的一组基,而且 $ dim(\mathcal{N}(\mathbf{A})) = n - r$,所以 $\rank{\mathbf{A}} + \dim(\mathcal{N}(\mathbf{A} ) ) = n$.
			\item 易知若 $ y = 0 $,那么对任意 $\mathbf{a}_i \in \{\mathbf{a}_1,\mathbf{a}_2,\ldots,\mathbf{a}_m\}$,,均有 $\mathbf{a}_i^{\top}\mathbf{y} = 0$. 反之,若对任意 $\mathbf{a}_i \in \{\mathbf{a}_1,\mathbf{a}_2,\ldots,\mathbf{a}_m\}$,,均有 $\mathbf{a}_i^{\top}\mathbf{y} = 0$,那么有 $\begin{pmatrix}
				a_1 & a_2 & \dots & a_m
			\end{pmatrix}^{\mathbf{T}} y = 0 $,假设 $ y \neq 0$,那么 $y = \begin{pmatrix}
				y1 & y2 & \dots & y_m
			\end{pmatrix}$,且至少存在一个分量不为 0 ,将矩阵乘法展开可知 $\mathbf{a}_1y_1 + \mathbf{a}_2y_2 + \dots + \mathbf{a}_my_m = 0$,由于 $\{\mathbf{a}_1,\mathbf{a}_2,\ldots,\mathbf{a}_m\}$ 是线性空间的一组基,根据定义一定有 $y_1 = y_2 = \dots = y_m = 0$,这与假设矛盾,所以 $ y = 0$.
		\end{enumerate}
	\end{enumerate}

\end{solution}

\newpage

\begin{exercise}[Projection \textnormal{30pts}]
	Let $C \subset \mathbb{R}^n$ be a closed convex set and $\mathbf{x} \in \mathbb{R}^n$. Define
	\begin{align*}
	    \proj{\mathbf{x}}{C} = \arg\min_{\mathbf{y} \in C}\| \mathbf{y} - \mathbf{x} \|_2.    
	\end{align*}
    We call $\proj{\mathbf{x}}{C}$ the projection of the point $\mathbf{x}$ onto the convex set $C$. 
    \begin{enumerate}
        \item Show that any finite dimensional space is convex.
        \item Let $\mathbf{v}_i \in \mathbb{R}^n$, $i=1,\ldots,d$ with $d\leq n$, which are linearly independent.
            \begin{enumerate}
		        \item For any $\mathbf{w}\in \mathbb{R}^n$, please find $\proj{\mathbf{w}}{\mathbf{v}_1}$, which is the projection of $\mathbf{w}$ onto the subspace spanned by $\mathbf{v}_1$.  
		        \item Please show $\proj{\cdot}{\mathbf{v}_1}$ is a linear map, i.e.,
		            \begin{align*}
		                \proj{\alpha\mathbf{u}+\beta\mathbf{w}}{\mathbf{v}_1}=\alpha\proj{\mathbf{u}}{\mathbf{v}_1} + \beta \proj{\mathbf{w}}{\mathbf{v}_1},
		            \end{align*}
		            where $\alpha,\beta\in\mathbb{R}$ and $\mathbf{w}\in\mathbb{R}^n$.
		        \item Please find the projection matrix corresponding to the linear map $\proj{\cdot}{\mathbf{v}_1}$, i.e., find the matrix $\mathbf{H}_1\in\mathbb{R}^{n\times n}$ such that
		            \begin{align*}
		                \proj{\mathbf{w}}{\mathbf{v}_1}=\mathbf{H}_1\mathbf{w}.
		            \end{align*}
		        \item Let $\mathbf{V}=(\mathbf{v}_1,\ldots,\mathbf{v}_d)$. 
		            \begin{enumerate}
		                \item For any $\mathbf{w}\in \mathbb{R}^n$, please find              $\proj{\mathbf{w}}{\mathbf{V}}$, which is the projection of $\mathbf{w}$ onto $\mathcal{C}(\mathbf{V})$, and the corresponding projection matrix $\mathbf{H}$. 
		                \item Please find $\mathbf{H}$ if we further assume that $\mathbf{v}_i^{\top}\mathbf{v}_j=0$, $\forall\,i\neq j$.
		            \end{enumerate}
	        \end{enumerate}
	   \item A matrix $\mathbf{P}$ is called a projection matrix if $\mathbf{P}\mathbf{x}$ is the projection of $\mathbf{x}$ onto $\mathcal{C}(\mathbf{P})$ for any $\mathbf{x}$.
	        \begin{enumerate}
	            \item Let $\lambda$ be the eigenvalue of $\mathbf{P}$. Show that $\lambda$ is either $1$ or $0$. (\emph{Hint: you may want to figure out what are the eigenspaces corresponding to $\lambda=1$ and $\lambda=0$, respectively.})
	            \item Show that $\mathbf{P}$ is a projection matrix if and only if $\mathbf{P}^2 = \mathbf{P}$.
	        \end{enumerate}
	\end{enumerate}
\end{exercise}

\begin{solution}
	\heiti
	\ \\
	\begin{enumerate}
		\item 有限维空间一般指有限维线性空间,对空间中的任意向量 $\mathbf{x}, \mathbf{y}$,都有 $\alpha\mathbf{x} + (1-\alpha)\mathbf{y}$ 仍然在空间中(线性空间定义),所以根据定义,任意有限维空间都是凸的。
		\item \ \\
		\begin{enumerate}
			\item $P_{v_{1}}(\mathbf{w})$ 是 $\mathbf{w}$ 在$\mathbf{v_1}$ 上的投影
			\begin{align*}
				P_{v_{1}}(\mathbf{w}) &=x \frac{\vec{v}_{1}}{\left|\vec{v}_{1}\right|}  \\ 
				x \ &= \ |\vec{w}| \cdot \cos <\vec{w}, \overrightarrow{v_{1}}>\\
				&=|\vec{w}| \cdot \frac{\vec{w} \cdot \vec{v}_{1}}{|\vec{w}| \cdot\left|\vec{v}_{1}\right|} \\ 
				&=\frac{\vec{w} \cdot \vec{v}_{1}}{\left|\vec{v}_{1}\right|} \\ 
				P_{v_{1}}(w) &= \frac{\vec{w} \cdot \vec{v}_{1}}{\left|\vec{v}_{1}\right|^2} \vec{v}_{1}
			\end{align*}
			\item \ \\
				\begin{align*} 
				P_{v_{1}}(\alpha \vec{u}+\beta \vec{w}) &=\frac{(\alpha \vec{u}+\beta \vec{w}) \vec{w}_{1}}{\left|\vec{w}_{1}\right|^2} \\ &=\frac{\alpha \vec{u} \cdot \vec{v}_{1}}{\left|\vec{v}_{1}\right|^2} \vec{v}_{1}+\frac{\beta \vec{w} \cdot \vec{v}_{1}}{\left|\vec{v}_{1}\right|^2} \vec{v}_{1} \\ &=\alpha P_{v_{1}}(\vec{u})+\beta P_{v_{1}}(\vec{w}) 
				\end{align*}
			\item 将上述结论写成矩阵形式得\\
				\begin{align*} 
				P_{v_{1}}(w) &=\frac{\vec{w} \cdot \vec{v}_{1}}{\left|\vec{v}_{1}\right|^{2}} \vec{v}_{1} \\ &=\frac{v_{1}^{\top} w}{v_{1}^{\top} v_{1}} v_{1} \\ &=\frac{\left.v_{1} (v_{1}^{\top} w\right)}{v_{1}^{\top} v_{1}} \\ &=\frac{v_{1} v_{1}^{\top}}{v_{1}^{\top} v_{1}} w 
				\end{align*}
				所以\\ 
				\begin{align*}
					\mathbf{H} = \frac{v_{1} v_{1}^{\top}}{v_{1}^{\top} v_{1}}
				\end{align*}
			\item 和上面相同的思路,设投影函数可以写成各个向量的线性组合
				\begin{align*}
				P_{\mathbf{v}}(\mathbf{w}) &=x_{1} \mathbf{v}_{1}+x_{2} \mathbf{v}_{2}+\cdots+x_{n} \mathbf{v}_{n} \\ &=\mathbf{V} \mathbf{x} 
				\end{align*}
				根据投影的定义,需要使得原来的点到投影之后的点的距离最小,那么向量 $\mathbf{w} - \mathbf{P}$ 一定垂直于 $\mathbf{V}$ 构成的子空间,也就与子空间中的每一组基垂直。于是有\\
				\begin{align*} 
				\mathbf{v}_{1}^{\top}(\mathbf{w}-\mathbf{P}) &=0 \\ \mathbf{v}_{2}^{\top}(\mathbf{w}-\mathbf{P}) &=0 \\ \vdots \\ \mathbf{v}_{d}^{\top}(\mathbf{w}-\mathbf{P}) &=0 \\ \Rightarrow\left(\begin{array}{c}{\mathbf{v}_{1}^{\top}}\\ \vdots \\ {\mathbf{v}_{d}^{\top}}\end{array}\right)(\mathbf{w}-\mathbf{P}) &=0 \\ \Rightarrow \quad \mathbf{V}^{\top}(\mathbf{w}-\mathbf{P})&=0 \\ 
				\end{align*}
				将 $\mathbf{P}$ 的表达式代入得\\
				\begin{align*} 
					\mathbf{V^{\top}(w-v x)=0} \\ \Rightarrow & \mathbf{V^{\top} w =V^{\top} V x} \\ \Rightarrow & \mathbf{x=\left(V^{\top} V\right)^{-1}VV^{\top} w}  \\ \Rightarrow & \mathbf{P=V\left(V^{\top} V\right)^{-1} V^{\top} w}  
				\end{align*}
				于是 $\mathbf{H} = \mathbf{V\left(V^{\top} V\right)^{-1} V^{\top}}$.
				\item $$
			v^{\top} v=\left(\begin{array}{c}{v_{1}^{\top}} \\ \vdots \\ {v_{d}^{\top}}\end{array}\right)\left(v_{1} \  v_{2} \cdots \  v_{d}\right)
			$$
			$\quad \begin{aligned} & \mathbf{=\left(\begin{array}{cccc}{v_{1}^{\top} v_{1}} \\ \  & {v_{2}^{\top} v_{2}} \\ \ & \ & \ddots \\ \ & & & {v_{d}^{\top} v_{d}} \\ \end{array}\right)} \\ \mathbf{\left(v^{\top} v\right)^{-1}} &=\left(\begin{array}{cccc} \frac{1}{{v_{1}^{\top} v_{1}}} \\ \  & \frac{1}{{v_{2}^{\top} v_{2}}} \\ \ & \ & \ddots \\ \ & & & \frac{1}{{v_{d}^{\top} v_{d}}} \\ \end{array}\right) \\ \mathbf{H=v\left(v^{\top} v\right)^{-1} v^{\top}} &=\left(\begin{array}{cccc} \frac{v_1v_1^{\top}}{{v_{1}^{\top} v_{1}}} \\ \  & \frac{v_2v_2^{\top}}{{v_{2}^{\top} v_{2}}} \\ \ & \ & \ddots \\ \ & & & \frac{v_dv_d^{\top}}{{v_{d}^{\top} v_{d}}} \\ \end{array}\right) \end{aligned}$
		\end{enumerate}
		\item 
		\begin{enumerate}
			\item 对于矩阵 $\mathbf{P}$,它的特征值和特征向量满足
				\begin{align*}
					\mathbf{P}\mathbf{x} = \lambda\mathbf{x}
				\end{align*}
			特征值和特征向量的几何解释为,$\mathbf{P}$ 看作一个线性变换的变换矩阵,$\mathbf{x}$ 是在变换之后保持不变的方向,这个方向上的向量只是长度伸缩了 $\lambda$ 倍。所以要求 $\mathbf{P}$ 的特征值,我们首先要寻找变换之后方向不变的特征向量。\\
			投影变换之后方向保持不变的向量有两类,一类是与投影平面垂直的向量,它们投影之后得到的向量是 $0$ 向量,即 $\mathbf{P}\mathbf{x} = 0 \cdot \mathbf{x} $,另一类是本身就在投影空间中的向量,投影之后得到的向量是它们本身,即 $\mathbf{P}\mathbf{x} = 1 \cdot \mathbf{x}$。综上,矩阵 $\mathbf{P}$ 的特征值 $\lambda$ 只能为 0 或者 1.
			\item \begin{enumerate}
				\item "$\Rightarrow$" \\
				\begin{align*}
					\mathbf{P^2} &= \mathbf{P P}\\
					& = \mathbf{V(V^{\top}V)^{-1}V^{\top}V(V^{\top}V)^{-1}V^{\top}}\\
					& = \mathbf{V(V^{\top}V)^{-1}(V^{\top}V)(V^{\top}V)^{-1}V^{\top}}\\
					& = \mathbf{V((V^{\top}V)^{-1}(V^{\top}V))(V^{\top}V)^{-1}V^{\top}}\\
					& = \mathbf{V(V^{\top}V)^{-1}V^{\top}}\\
					& = \mathbf{P}
				\end{align*}
				\item "$\Leftarrow$" \\
				设 $\mathbf{y}$ 是 $\mathbf{x}$对于子空间的投影,那么有 
				\begin{align*}
					\mathbf{P(x)} & =\mathbf{y}\\
					\mathbf{P(y)} & = \mathbf{P(P(x))}\\
					& = \mathbf{P^2(x)} \\
					& = \mathbf{P(x)}\\
					& = \mathbf{y}
				\end{align*}
				这说明 $\mathbf{y}$ 在子空间中,而且 $P$ 是投影矩阵。
					
			\end{enumerate}
		\end{enumerate}
	\end{enumerate}
\end{solution}

\newpage

\begin{exercise}[ \textnormal{5pts}] 
	Let $\mathbf{x}\in \mathbf{R}^n$. Find the gradients of the following functions.
	\begin{enumerate}
	    \item $f(\mathbf{x}) = \mathbf{a}^{\top}\mathbf{x}$.
	    \item $f(\mathbf{x}) = \mathbf{x}^{\top}\mathbf{x}$.
	    \item $f(\mathbf{x})=\| \mathbf{y} - \mathbf{A}\mathbf{x} \|_2^2$, where $\mathbf{A}\in\mathbb{R}^{m\times n}$.
	\end{enumerate}
\end{exercise}

\begin{solution}
	\heiti
	\ \\
	\begin{enumerate}
		\item $\frac{\mathbf{d}f(\mathbf{x})}{\mathbf{d}x} = \mathbf{a}$
		\item $\frac{\mathbf{d}f(\mathbf{x})}{\mathbf{d}x} = 2\mathbf{x}$
		\item $\frac{\mathbf{d}f(\mathbf{x})}{\mathbf{d}x} = \frac{\mathbf{d}f(\mathbf{x})}{\mathbf{d}(\mathbf{y} - \mathbf{A}\mathbf{x})}  \frac{\mathbf{d}(\mathbf{y} - \mathbf{A}\mathbf{x})}{\mathbf{d}x} = - 2 \mathbf{A}^\mathbf{T} (\mathbf{y} - \mathbf{A}\mathbf{x})$
	\end{enumerate}
\end{solution}
\newpage


\begin{exercise}[Second-order sufficient optimality conditions \textnormal{10pts}]
	Suppose that $f:\mathbb{R}^n\rightarrow\mathbb{R}$ is twice differentiable at $\mathbf{x}$. If $\nabla f(\mathbf{x})=0$ and the Hessian matrix $\mathbf{H}(\mathbf{x})$ is positive definite, then $\mathbf{x}$ is a strict local minimum.
	\begin{enumerate}
		\item Show the above result by contradiction.
		\item Show the result by NOT using contradiction. [\emph{Hint: you may need eigen-decomposition.}]
	\end{enumerate}
\end{exercise}

\begin{solution}
	\heiti
	\ \\
	\begin{enumerate}
		\item 设 $\mathbf{x}$ 不是严格局部最小值,那么存在 $\mathbf{x_0}$ 使得 $f(\mathbf{x_0}) \leq f(\mathbf{x})$,对 $f(\mathbf{x_0})$ 做泰勒展开可得
			\begin{align*}
				f(\mathbf{x_0}) = f(\mathbf{x}) + \nabla f(\mathbf{x})(\mathbf{x} - \mathbf{x_0}) + \frac{1}{2}(\mathbf{x_0} - \mathbf{x})^\mathbf{T}\mathbf{H}(\mathbf{x})(\mathbf{x_0} - \mathbf{x}) + o((\mathbf{x_0} - \mathbf{x})^2)
			\end{align*}
		根据题目条件 $\mathbf{H}(\mathbf{x})$ 正定可知 $\frac{1}{2}(\mathbf{x_0} - \mathbf{x})^\mathbf{T}\mathbf{H}(\mathbf{x})(\mathbf{x_0} - \mathbf{x}) > 0$,再根据$\nabla f(\mathbf{x})=0$ 可得
			\begin{align*}
				f(\mathbf{x_0}) > f(\mathbf{x})
			\end{align*}
		与假设矛盾,所以$\mathbf{x}$ 处是 $f(\mathbf{x})$ 局部严格最小值。
		\item 对于 $\mathbf{x}$ 附近的任意一点 $\mathbf{x}_0$,在此处对 $f(\mathbf{\mathbf{x}_0})$ 做泰勒展开可得
			\begin{align*}
				f(\mathbf{x_0}) = f(\mathbf{x}) + \nabla f(\mathbf{x})(\mathbf{x} - \mathbf{x_0}) + \frac{1}{2}(\mathbf{x_0} - \mathbf{x})^\mathbf{T}\mathbf{H}(\mathbf{x})(\mathbf{x_0} - \mathbf{x}) + o((\mathbf{x_0} - \mathbf{x})^2)
			\end{align*}
			因为 $\mathbf{H}(\mathbf{x})$ 是正定的,所以 $\frac{1}{2}(\mathbf{x_0} - \mathbf{x})^\mathbf{T}\mathbf{H}(\mathbf{x})(\mathbf{x_0} - \mathbf{x}) > 0$,又因为 $\nabla f(\mathbf{x})=0$,可以看出
				\begin{align*}
					f(\mathbf{x_0}) > f(\mathbf{x})
				\end{align*}
			故 $\mathbf{x}$ 处是 $f(\mathbf{x})$ 局部严格最小值。
	\end{enumerate}

\end{solution}
\newpage

\begin{exercise}[Identically independently distributed \textnormal{10pts}]
	Suppose that the training samples $\{(\mathbf{x}_i,y_i)\}_{i=1}^n$ are i.i.d.. show that
	\begin{align*}
	    p(\mathbf{x}_1,\ldots,\mathbf{x}_n)=\prod_{i=1}^np(\mathbf{x}_i).
	\end{align*}
	
\end{exercise}

\begin{solution}
	\heiti
	\ \\
	根据独立的定义,n 个样本相互独立,那么它们联合概率分布等于分别的概率分布乘积,即
	\begin{align*}
	    p(\mathbf{x}_1,\ldots,\mathbf{x}_n)=\prod_{i=1}^np(\mathbf{x}_i).
	\end{align*}
	于是得证。
\end{solution}
\newpage


\begin{exercise}[First-order condition \RNum{2} \textnormal{5pts}]
	Suppose that $f$ is continuously differentiable. Prove that $f$ is convex if and only if $\dom f$ is convex and
	\begin{align*}
	    \langle \nabla f(\mathbf{x}) - \nabla f(\mathbf{y}) , \mathbf{x} - \mathbf{y} \rangle \geq 0.
	\end{align*}
	
\end{exercise}

\begin{solution}
	\heiti
	\ \\
	\begin{enumerate}
		\item "$\Rightarrow$"\\
			当 $f(\mathbf{x})$ 是凸函数时,有 
			\begin{align*}
				f(\mathbf{y}) \geq f(\mathbf{x}) + \langle \nabla f(\mathbf{x}) , \mathbf{y} - \mathbf{x} \rangle.\\
				f(\mathbf{x}) \geq f(\mathbf{y}) + \langle \nabla f(\mathbf{y}) , \mathbf{x} - \mathbf{y} \rangle.\\
				\Rightarrow f(\mathbf{y}) \geq f(\mathbf{y}) + \langle \nabla f(\mathbf{y}) , \mathbf{x} - \mathbf{y} \rangle + \langle \nabla f(\mathbf{x}) , \mathbf{y} - \mathbf{x} \rangle.\\
				\Rightarrow \langle \nabla f(\mathbf{x}) , \mathbf{x} - \mathbf{y} \rangle - \langle \nabla f(\mathbf{y}) , \mathbf{x} - \mathbf{y} \rangle \geq 0.\\
				\Rightarrow \langle \nabla f(\mathbf{x}) - \nabla f(\mathbf{y}) , \mathbf{x} - \mathbf{y} \rangle \geq 0.
			\end{align*}
		\item "$\Leftarrow$"\\
		首先证明如果函数 $f(\mathbf{x})$ 的一阶导数递增,那么该函数为凸函数。\\
		由拉格朗日中值定理,存在$\xi_1 \subset (\theta x+(1-\theta) y, y), \xi_2 \subset (x, \theta x+(1-\theta) y)$ 使得
		$$
		\begin{array}{l}{f(y)-f(\theta x+(1-\theta) y)=f^{\prime}\left(\xi_{1}\right)(y-(\theta x+(1-\theta) y)) \cdots (1)} \\ {f(x)-f(\theta x+(1-\theta) y)=f^{\prime}\left(\xi_{2}\right)\left(y-(\theta x+(1-y) y)) \cdots (2)\right.} \\ {\frac{(1)}{\theta}+\frac{(2)}{1-\theta} \Rightarrow} \\ {\frac{f(y)-f(\theta x+(1-\theta) y)}{\theta}+\frac{f(x)-f(\theta x+(1-\theta) y)}{1-\theta}=\left(f^{\prime}\left(\xi_{2}\right)-f^{\prime}\left(\xi_{1}\right)\right)(y - x) \cdots (3)} \\Since {f^{\prime}\left(\xi_{1}\right) \geqslant f^{\prime}\left(\xi_{2}\right)}, so (3) \geq 0 \\ {\Rightarrow f(\theta x+(1-\theta) y) \leqslant \theta f(x)+(1-\theta) f(y)}\end{array}
		$$
		构造函数 $  F(\lambda) = f(\mathbf{x} + \lambda(\mathbf{y}-\mathbf{x})$,则凸函数条件可以转化为:
			\begin{align*}
				\lambda f(\mathbf{x})+(1-\lambda) f(\mathbf{y}) \geqslant f(\lambda \mathbf{x}+(1-\lambda) \mathbf{y}) \\ \Leftrightarrow \lambda F(0)+(1-\lambda) F(1) \geqslant F(1-\lambda)
			\end{align*}
		只需证明 $F(\lambda)$ 在 $[0, 1]$ 上是凸函数即证明 $F^{\prime}(\lambda)$ 递增。\\
		由于 ${\left.\quad F^{\prime}(x)=\langle\nabla f(x+\lambda ( y-x), y-x\right\rangle}$\\
		取 $\lambda_1 > \lambda_2 $,只需证明\\
		$\begin{array}{l} { F^{\prime}(\lambda_1) \geqslant F^{\prime}\left(\lambda_{2}\right)} \\ {\Leftrightarrow \langle \nabla f(\mathbf{x}+\lambda,(\mathbf{y}-\mathbf{x})), \quad \mathbf{y}-\mathbf{x} \rangle\geqslant\left\langle \nabla f\left(\mathbf{x}+\lambda_{2}(\mathbf{y}-\mathbf{x})\right), \mathbf{y}-\mathbf{x}\right\rangle} \\ {\Leftrightarrow \langle \nabla f(\mathbf{x}+\lambda,(\mathbf{y}-\mathbf{x})),\left(\lambda_1-\lambda_{2}\right)(\mathbf{y}-\mathbf{x}) \rangle \geqslant\left\langle\nabla f\left(\mathbf{x}+\lambda_{2}(\mathbf{y}-\mathbf{x}),(\lambda_1 - \lambda_2)(\mathbf{y} - \mathbf{x})\right \rangle \right.} \\ {Let \  \mathbf{m} = \quad \mathbf{x}+\lambda_{1}(\mathbf{y}-\mathbf{x})} \\ {Let \  \mathbf{n} = \quad \mathbf{x}+\lambda_{2}(\mathbf{y}-\mathbf{x})} \\ {\Leftrightarrow \langle\nabla f(\mathbf{m}), \mathbf{m}-\mathbf{n}\rangle \geqslant\langle\nabla f(\mathbf{n}), \mathbf{m}-\mathbf{n}\rangle} \\ {\Leftrightarrow \langle\nabla f(\mathbf{m}) - \nabla f(\mathbf{n}), \mathbf{m}-\mathbf{n}\rangle \geqslant 0} \end{array}$\\
		所以得证。
			
	\end{enumerate}

\end{solution}


%%%%%%%%%%%%%%%%%%%%%%%%%%%%%%%%%%%%%%%%%%%%%%%%%%%%%%%%%%%%%%%%

\end{document}
